%gibt an: Papierformat, Schriftgr��e
\documentclass[a4paper,german,12pt]{article}
\setlength{\parskip}{0.2cm}
\setlength{\parindent}{0cm}
%-----------------------------------------------------------------------------------------------------------------------------
%�bersetzung von E in D
\usepackage{ngerman}
%Einstellung der Randabst�nde
\usepackage[lmargin={2.5cm},rmargin={2.5cm},tmargin={1.5cm},bmargin={2.5cm}]{geometry}
%zur Einbindung von Graphiken
\usepackage{graphicx}
%Bearbeitung von Kopf- und Fusszeile
\usepackage{fancyhdr}
%Schriftart
\usepackage{helvet}
%stellt unabh�ngige Textmarken zu Verf�gung
\usepackage{extramarks}
%aktiviert eine Umgebung in der der Mathematikmodus aktiv ist
\usepackage{amsmath}
%aktiviert eine Umgebung in der der Mathematikmodus aktiv ist
\usepackage{amsthm}
%aktiviert eine Umgebung in der der Mathematikmodus aktiv ist
\usepackage{amssymb}
%aktiviert Hyperlinks
\usepackage{hyperref} 
%Stellt das Eurozeichen � zu Verf�gung
\usepackage[right]{eurosym}
%�bersetzt die Tastatureingaben f�r LaTex
%\usepackage[latin1]{inputenc}
\usepackage{fancybox}
\usepackage{xcolor}
\usepackage{color}
\usepackage{float}
\usepackage{framed}
\usepackage{url}
\usepackage[ansinew]{inputenc}

%\usepackage[square]{natbib}

%Weitere
\usepackage{bibgerm} %Bibliothek

\usepackage{booktabs}
\usepackage{tabularx} %Tabellen, die sich der Seitenbreite anpassen

\usepackage{multirow} % Verbundene Zellen in Tabellen

\usepackage{rotating} % Quergestellte Tabellen
\usepackage{rotfloat}

\usepackage[bottom]{footmisc} %Fu�noten immer am Ende der Seite

%\usepackage[T1]{fontenc}
\usepackage[final]{microtype}

%Definitionen

\newtheoremstyle{mystyle}% name
{10pt}% Space above
{10pt}% Space below
{\itshape}% Body font
{}% Indent amount: 
{\bfseries}% Theorem head font
{:}% Punctuation after theorem head
{0.5em}% Space after theorem head
{\thmname{#1}\thmnumber{ #2}:\thmnote{ #3}}% Theorem head 

\theoremstyle{mystyle}% default
\newtheorem{definition}{Definition}
\numberwithin{equation}{section}
\renewcommand{\proofname}{Beweis}

%Paragraphen

\newcommand{\myparagraph}[1]{\paragraph{}\mbox{}\\}


%-----------------------------------------------------------------------------------------------------------------------------------------
\renewcommand{\baselinestretch}{1.2}
\fancyhead[LO]{\slshape \small \firstleftmark}
\fancyhead[RO]{\normalsize\thepage} \fancyfoot{}
\begin{document}
%-----------------------------------------------------------------------------------------------------------------------------------------
%Titelseite

\begin{titlepage}
% font / Schriftart
%------------------	
\begin{figure}[htbp]
  %\centering
  \begin{minipage}[t]{0.4\textwidth} 
    \includegraphics[scale=0.75]{0_Logos/kit.jpg}  
    
  \end{minipage}
  \hfill
  \begin{minipage}[b]{0.3\textwidth} 
	\begin{flushleft}
		\includegraphics[scale=0.09]{0_Logos/aifb.png}     
	\end{flushleft}
	\begin{flushleft}
		\tiny\textbf{Institut f�r Angewandte Informatik}
		\textbf{und Formale Beschreibungsverfahren}
	\end{flushleft}
  \end{minipage}
\end{figure}

\vspace*{1.5cm}
\leftskip=4cm
		\textbf{{\Huge %TODO
		Titel der Abschlussarbeit}\\
		%{\Huge
		%[auch mehrzeilig m�glich]}
		}
		\vspace*{1.5cm}\\
		{\Large Bachelorarbeit}\\
		{\Large von} \\
		{\Large Marius Wodtke}
		\vspace*{3.5cm}\\
		an der Fakult�t f�r Informatik
		\vspace*{0.5cm}\\
		In dem Studiengang\\
		Informatik (B.Sc.)
		\vspace*{1.5cm}\\
		eingereicht am 01.01.01 %TODO Datum
		beim\\
		Institut f�r Angewandte Informatik\\
		und Formale Beschreibungsverfahren\\
		des Karlsruher Instituts f�r Technologie
		\vspace*{2.5cm}\\
		Referent: Prof. Dr. Hartmut Schmeck
		\\
		Betreuer: Kaibin Bao
		\vspace*{1.5cm}\\
		{\footnotesize KIT -- Universit�t des Landes Baden-W�rttemberg und}\\
		{\footnotesize nationales Forschungszentrum in der Helmholtz-Gemeinschaft}\\
		\hfill

\end{titlepage}


\thispagestyle{empty}\cleardoublepage

\mbox{}\thispagestyle{empty}
\cleardoublepage

\mbox{}\thispagestyle{empty}

\vspace*{1cm}

{\Large \textbf{Eidesstattliche Erkl�rung}} 

\bigskip

Ich versichere hiermit wahrheitsgem��, die Arbeit und alle Teile daraus selbst�ndig angefertigt, alle benutzten Hilfsmittel vollst�ndig und genau angegeben und alles kenntlich gemacht zu haben, was aus Arbeiten anderer unver�ndert oder mit Ab�nderung entnommen wurde.\\

\vspace{1cm}
 
\textit{Karlsruhe, den DATUM} \hspace{4cm} \textit{NAME} \\

\thispagestyle{empty}\cleardoublepage

\mbox{}\thispagestyle{empty}
\cleardoublepage
\rmfamily \pagestyle{fancy} \setcounter{secnumdepth}{4}


\pagenumbering{roman}
\setcounter{page}{3} 
\tableofcontents
\newcounter{roemisch} 
\setcounter{roemisch}{\value{page}}
\markboth{ABK�RZUNGSVERZEICHNIS}{Abk�rzungsverzeichnis}
\section*{Abk�rzungsverzeichnis}
\begin{table}[h]
    \begin{tabular}{rl}
    NILM & Non-intrusive load monitoring\\
    KNN & K�nstliches Neuronale Netz\\
    \end{tabular}%
\end{table}%

\cleardoublepage
\listoffigures
\cleardoublepage
\listoftables
\cleardoublepage

\setcounter{page}{2} \pagenumbering{arabic}

\interfootnotelinepenalty=10000 % Keine Seitenumbr�che bei Fu�noten

 % % % % % % % % % % % % %INHALTE % % % % % % % % % % % % % %
 \section{Einleitung}
\label{Einleitung}

%\subsection{Motivation}
%\label{Motivation}
	Energie zu sparen ist seit Langem ein Ziel der Umweltpolitik, dennoch steigt der Energieverbrauch in Industriel"andern kontinuierlich und es werden nach wie vor Wege gesucht, diesen zu senken. 
	In den USA verursachen private Haushalte 40\% des CO2-Aussto{\ss}es \cite{vandenbergh2008individual} und stehen deshalb im Fokus vieler Programme zum Energiesparen. \\
	Je nach Studie besteht f"ur die Haushalte ein Einsparungspotential von bis zu 20\% \cite{armel2013disaggregation} f"ur dessen Energieverbrauch, Fischer \cite{fischer2008feedback} untersucht mehrere dieser Studien und beschreibt ein durchschnittliches Einsparungspotential von 5-12\%. 
	Um dieses Einsparungspotential auszunutzen, muss der Nutzer regelm"a{\ss}ig und m"oglichst genau "uber seinen Verbrauch aufgekl"art werden. 	%TODO Graphik Armel
	\begin{figure}[ht]
\includegraphics[width=\textwidth]{1_Grafiken/fig1armel.jpg}
	\caption[Einsparungpotentiale nach Ma{\ss}nahme]{Grafik zu Einsparpotetialen je nach getroffener Aufkl"arungsma{\ss}nahme aus \cite{armel2013disaggregation}}
\label{potentiale}
\end{figure}

	Er kann sich oft nicht mit seinem Verbrauch identifizieren, weil dieser intransparent und durch die langen Rechnungsintervalle nicht pr"asent ist. \\
	Damit der Nutzer anf"angt sich selbst zu kontrollieren, muss er begreifen, dass sich sein Verhalten auf seinen Verbrauch auswirkt und er ihn auch durch die gezielte Ver"anderungen seines Handelns senken kann \cite{fischer2008feedback}.\\
	Non-intrusive load monitoring (NILM) bietet nun die Chance dem Nutzer eine regelm"a{\ss}ige, detaillierte R"uckmeldung zu seinem Energiekonsum zu geben, Kolter und Matthew \cite{kolter2011redd} beschreibt NILM als die Aufgabe aus einem, f"ur den gesamten Haushalt messenden Stromz"ahler, R"uckschl"usse "uber die elektrische Last einzelner Ger"ate zu ziehen.
	Ein solches System wird dem Konsumenten helfen, verschwenderische Verhaltensweisen und Ger"ate zu identifizieren ohne den Haushalt mit vielen digitalen Z"ahlern f"ur die individuellen Ger"ate ausr"usten zu m"ussen, wie es derzeit z.B. mit sogenannten \textit{Energiekostenmessger"aten} auf Steckerbasis praktiziert wird.

\subsection{Fragestellung}
\label{Fragestellung}
 	Viele Systeme zur Disaggregation oder zu verschiedenen Vorhersage-Aufgaben benutzen Ans"atze des Maschinellen Lernens. Sie ben"otigen annotierte (gelabelte) Trainingsdaten, um robuste Modelle zu erzeugen. Annotieren bedeutet in diesem Zusammenhang, die Daten mit Metadaten wie etwa dem Ger"atezustand zu versehen. Mehr Trainingsdaten f"uhren in der Regel zu besseren Modellen, die Annotation ist allerdings sehr aufw"andig, weil sie manuell erfolgen muss. Der Ansatzpunkt dieser Arbeit ist, das Problem der Disaggregation auf ein Ger"at zu reduzieren und so zu vereinfachen. Nun kann man ein robustes Modell mit nur wenigen Trainingsdaten erstellen und mit diesem dann eine gro{\ss}e Menge an Daten schnell annotieren. Diese gelabelten Daten sollen dann wiederum als Trainingsdaten f"ur schwierigere Aufgaben verwendet werden. 
	Aufgabe dieser Bachelorarbeit ist es ein System zu entwickeln, welches in der Lage ist die Zust"ande von ausgew"ahlten Ger"aten in einem disaggregierten Lastgang zu klassifizieren und so eine Segmentierung f"ur diese Ger"ate zu erstellen. Dies beinhaltet sowohl die n"otige Vorverarbeitung der Daten sowie eventuelle nachtr"agliche Formatierungen. Die eigentliche Klassifikation soll mit K"unstlichen Neuronalen Netzen (KNN) stattfinden, dabei sollte eine m"oglichst hohe Akkurarit"at erreicht werden, da Systeme, die mit den resultierenden Daten trainiert werden keine M"oglichkeit haben Fehler, die bereits in den Trainingsdaten sind, zu korrigieren. 
Au{\ss}erdem soll versucht werden aus den klassifizierten Daten wieder einen Lastgang zu erstellen. Dies soll die M"achtigkeit der Segmentierung der Ger"atezust"ande untersuchen und feststellen ob man zuverl"assig den Lastgang eines Ger"ats vorhersagen kann, wenn seine Zustandsfolge im Voraus bekannt ist. 

\subsection{Weiterer Aufbau}
\label{Weiterer Aufbau}
	Im folgenden Kapitel~\ref{Datensatz} werden die verwendeten, disaggregierten Energiedatens"atze beschrieben. Insbesondere werden die gemessenen Werte und die daraus berechenbaren Werte untersucht. 
	Kapitel~\ref{Vorverarbeitung} gibt zun"achst einen "Uberblick "uber die aktuelle Forschung im Bereich Vorverarbeitung der Daten und Feature-Auswahl, anschlie{\ss}end werden die f"ur diese Arbeit verwendeten Vorverarbeitungsschritte und Features erl"autert.
	In Kapitel~\ref{Klassifizierung} soll schlie{\ss}lich die eigentliche Klassifizierung beschrieben werden, hier werden insbesondere das verwendete Netz und die Trainingsmethoden besprochen. Auch hier wird es einen kurzen "Uberblick "uber die aktuelle Forschung geben.
	In Kapitel~\ref{Generierung} wird die Generierung eines Lastprofils aus dem Zustandsprofil erl"autert. Es wird ein Ansatz aus der aktuellen Forschung beschrieben und auf dieses Problem adaptiert. 
	In Kapitel~\ref{Evaluation} findet die Evaluation statt, hier werden verschiedene Aufgaben ausgef"uhrt und ausgewertet. 
	Am Schluss steht das Fazit in Kapitel~\ref{fazit}, in dem die Ergebnisse zusammengefasst und mit urspr"ungliche Fragestellung verglichen werden. 
 
 \section{Datensatz}
\label{Datensatz}

Die verwendeten Daten stammen aus dem \textit{KIT Energy Smart Home Lab (ESHL)}. Das ESHL ist ein $60 m^2$ Apartment mit intelligenten Haushaltsger"aten und soll einen zuk"unftigen Haushalt simulieren. Die Daten des ESHL wurden "uber fast 3 Jahre zwischen dem 22.08.2011 und dem 31.07.2014 aufgezeichnet, weisen aber einige L"ucken auf. F"ur die Messungen wurden mehrere Stromz"ahler verwendet, bei Gro{\ss}verbrauchern wie etwa dem Boiler wurden dabei alle drei Phasen gemessen, sie sind durch die Anschlussnummer des Stromz"ahlers (Meter) in den Daten identifizierbar. Bei den restlichen Ger"aten wurde nur je eine Phase gemessen, das hei{\ss}t, dass  3 unterschiedliche Ger"ate an einem Stromz"ahler angeschlossen sind. Sie sind durch die Nummer des Stromz"alers in Kombination mit Nummer des verwendeten Anschlusses (Port) identifizierbar. Wie viele Phasen zur Messung eines Ger"ats verwendet wurden l"asst sich dem Kommunikationsgateway (Controller) entnehmen.
Tabelle~\ref{uuid} zeigt die beschriebene Zuordnung von Ger"aten zu Stromz"ahler und Anschluss.
\begin{table}[h]
\begin{tabular}{l|l|l|l|l|l}
UUID & Name & Klassifikation & Controller & Meter & Port \\
\hline
...-5602c0a80114 & DRYER & APPLIANCE & 1 & 3 & 0 \\
...-5604c0a80114 & WASHINGMACHINE & APPLIANCE & 1 & 7 & 2
\end{tabular}
\caption["Ubersicht Ger"atezuordnung]{Beispiele f"ur Ger"atezuordnung, die Waschmaschine ist an Meter 7 Port 2 angeschlossen.}
\label{uuid}
\end{table}



\subsection{Gemessene Daten}
\label{Gemessene Daten}

Die Daten des \textit{KIT Energy Smart Home Lab} besitzen folgendes Format (Tabelle~\ref{formateshl}). \\
\begin{table}[h]
\begin{tabular}{l|l|p{1cm}|p{1cm}|p{1cm}|p{1cm}|p{1cm}|p{1.2cm}|p{2cm}}
Unixtime & UUID & Con-troller & Meter & Port & Span-nung & Strom-st"arke & Wirk-leistung & Z"ahlerstand \\
\hline
1382291846 & ...-5604c0a80114 & 1 & 7 & 2 & 218.5 & 8.96 & 1956 & 145550  \\
1315269463 & ...-5602c0a80114 & 1 & 3 & 0 & 223.8 & 13.16 & 2950 & 38300 \\
1315269465 & ...-5602c0a80114 & 1 & 3 & 0 & 223.9 & 12.5 & 2796 & 38300
\end{tabular}
\caption[Format ESHL Daten]{Format ESHL Datensatz.}
\label{formateshl}
\end{table}

Controller, Meter und Port werden, wie oben beschrieben, verwendet, um einen Datenpunkt einen eindeutigen Ger"at zuzuordnen. Hier ist zu beachten, sich diese Werte immer nur gemeinsam verwendet werden d"urfen, nur so ist eine eindeutige Identifizierung eines Ger"ates m"oglich.
Die Spalte mit der UUID-Zuordnung wird ignoriert, denn sie enth"alt ebenfalls die Information um welches Ger"at es sich handelt und ist somit redundant zu Controller, Meter und Port. \\
Ein Datenpunkt enth"alt die gemessene Spannung, die Stromst"arke und die Wirkleistung. Es wird zus"atzlich die aggregierte Wirkleistung (Z"ahlerstand) gemessen, welche aber zun"achst ignoriert wird, da sie sich aus der Wirkleistung berechnen l"asst. \\
Eine Besonderheit stellt das Aufzeichnungsintervall dar. Die Stromz"ahler erzeugen jede Sekunde einen neuen Datenpunkt f"ur jeden Anschluss, dieser wird aber nur gespeichert, wenn sich die Wirkleistung im Vergleich zum letzten gespeicherten Punkt um mindestens 5 Watt Wirkleistung unterscheidet. So werden insbesondere lange \textit{Off} Phasen auf einen Datenpunkt komprimiert, es gehen aber auch kleinere Schwankungen in der Wirkleistung von sehr verbrauchsarmen Ger"aten verloren.


\subsection{Berechnete Daten}
\label{Berechnete Daten}

Die Messung von Spannung und Stromst"arke erlaubt es eine Vielzahl von Energiewerten zu berechnen, in diesem Abschnitt werden die Schein- und Blindleistung vorgestellt, das Kapitel zur Vorverarbeitung wird zus"atzlich eine Form der Normalisierung vorstellen. 
Die Scheinleistung $S$ ist das Produkt aus Spannung $U$ und Stromst"arke $I$ und ist die gesamte, aus dem Netz gezogene Leistung:\\ $S = U * I$\\[0.5cm]
Zur Berechnung der Blindleistung $Q$ werden Schein- und Wirkleistung $P$ genutzt, die Blindleistung ist dabei Leistung, die aus dem Netz gezogen wird ohne tats"achlich genutzt zu werden:\\ $Q = \sqrt{S^2 - P^2}$ \\


\subsection{Die Waschmaschine}
\label{Die Waschmaschine}

Die Waschmaschine ist ein besonders interessantes Ger"at f"ur die Klassifikation, denn sie hat nicht nur einen \textit{On-} und einen \textit{Off-Zustand}, sondern einen Motor, der mit verschiedenen Drehzahlen laufen kann, ein Heizelement und eine Pumpe. F"ur die Waschmaschine existieren au{\ss}erdem einige annotierte Waschg"ange (Tabelle \ref{profile01458}), die als Trainingsdaten verwendet werden k"onnen.\\
\begin{table}[h]
\begin{tabular}{l|p{3.5cm}|p{3.5cm}|l|l}
Profil & Startzeit \newline (Unix time stamp) & Endzeit \newline (Unix time stamp) & Dauer (Sekunden) & Datum \\
\hline
Profil 0 & 1382290611 & 1382296244 & 5633 & 20.10.2013 \\
Profil 1 & 1382363246 & 1382369468 & 6222 & 21.10.2013 \\
Profil 4 & 1382900551 & 1382905960 & 5409 & 27.10.2013 \\
Profil 5 & 1383173215 & 1383180425 & 7210 & 30.10.2013 \\
Profil 8 & 1384066710 & 1384072260 & 5550 & 10.11.2013
\end{tabular}
\caption["Ubersicht Trainingsdaten]{"Ubersicht "uber die annotierten Waschg"ange, die als Trainingsdaten verwendet werden.}
\label{profile01458}
\end{table}\\
Die Waschg"ange sind mit sechs verschiedenen Klassen Sigma 0 bis Sigma 5 annotiert, Sigma 0 ist der \textit{Off-Zustand}. Sigma 1 ist ein normaler Schleudervorgang, hier sind Phasen mit 200-350 Watt und dazwischen kurze Pausen mit ca. 4 Watt Verbrauch typisch.
Sigma 2 ist ein Heizvorgang, dieser kennzeichnet sich durch einen Verbrauch "uber 1700 Watt.
Sigma 3 und Sigma 4 sind Pumpvorg"ange und werden durch Verbrauchspitzen von wenigen Sekunden charakterisiert. Unterschieden werden sie anhand der H"ohe der Spitze, ein Vorgang Sigma 4 folgt immer auf einen Vorgang Sigma 3 mit einem Abstand von 50 bis 80 Sekunden.
Sigma 5 ist das Ausschleudern, es ist ebenfalls ein Schleudervorgang, der aber im Gegensatz zu Sigma 1 ohne Pausen ausgef"uhrt wird. \\

Grafik~\ref{typWasch} veranschaulicht die Wertebereiche und Lage der Klassen innerhalb eines Waschgangs. Die Klassen sind dabei wie in der Legende beschrieben farbig markiert, da die dunkelblau  gef"arbte Klasse 0 typischerweise einen Verbrauch von 0 Watt aufweist verl"auft der Graph zu diesen Zeitpunkten auf der x-Achse. Die rot gekennzeichneten Abschnitte der Heizvorg"ange sind in der ersten H"alfte des Waschgangs zu finden, hier wird das Wasser zun"achst aufgew"armt und muss dann mehrfach nachgeheizt werden um die Temperatur zu halten. Die hellblau und lila markierten Abschnitte der Pumpvorg"ange befinden sich in der 2. H"alfte des Waschgangs. Sie sind sehr kurz und folgen in regelm"a{\ss}igen Abst"anden aufeinander. Der Ausschleudervorgang von Klasse 5 ist nahe dem Ende des Waschgangs zu finden, er ist gelb gekennzeichnet.
\begin{figure}[ht]
\includegraphics[width=\textheight , angle=90]{1_Grafiken/classes0.png}
	\caption[Typischer Waschgang, farbig annotiert]{Ein typischer Waschgang, je nach Klasse wird die Wirkleistung in unterschiedlichen Farben dargestellt}
\label{typWasch}
\end{figure}


 \section{Vorverarbeitung}
\label{Vorverarbeitung}


\subsection{State-of-the-Art}
\label{State-of-the-Art}
	Urspr"unglichstes Merkmal um die Zust"ande von Ger"aten zu unterscheiden ist die Wirkleistung.\\
	Sie gibt die elektrische Leistung an, die von einem Ger"at in andere Leistungen (z.B. W"arme oder Bewegung) umgewandelt werden kann. \\ %TODO Verweis finden
	Oft wird die Wirkleistung durch gr"obere Quantisierung oder Normalisierung gegl"attet, \cite{hart1992nonintrusive} verwendet hierzu folgende 		Formel:\\ 
	$P_{norm} (t) = (\frac{Nennspannung}{V(t)})^2*P(t)$ mit P := Wirkleistung in Watt und V := Spannung in Volt, die Nennspannung betr�gt in Europa 230 Volt.\\
	Die Normalisierung bietet den Vorteil, dass sie nicht verlustbehaftet ist und wird deshalb h"aufig verwendet, wenn die Spannung gemessen wurde.\\
	Mit der Wirkleistung allein ist man jedoch nicht in der Lage Ger"ate mit sehr "ahnlichem Verbrauch zu unterscheiden, ein Beispiel w"aren hier ein 2kW Motor und ein 2kW Heizelement.\\
	 Diese lassen sich mit Hilfe der Blindleistung, welche nicht in andere Leistungen umgewandelt wird, unterscheiden, weil der Motor als induktiver Verbraucher mehr Blindleistung als das Heizelement aus dem Stromnetz zieht. \\ %TODO Grafik Hart und WaMa KIT
	 Nachteil hier ist jedoch, dass die Blindleistung (und auch die Spannung) extra gemessen werden muss, die Ber"ucksichtigung dieses Features kostet also Geld \cite{zeifman2011nonintrusive}.
	 Vergleicht man jedoch eine 60W Gl"uhbirne und einen Laptop mit einem 60W Netzteil, dann stellt man fest, dass diese sich auch unter Hinzunahme der Blindleistung kaum unterscheiden lassen \cite{laughman2003power}. \\
	 Hier hilft die Betrachtung von "micro level" Features. \\
	 Hier werden Wellenformen und die harmonischen Komponenten des Frequenzspektrums verwendet um Verbraucher zu unterscheiden, \cite{laughman2003power} zeigt hier z.B., dass sich Gl�hbirne und Netzteil in der 3. harmonischen Komponente deutlich unterscheiden.
	 Zus"atzlich ver"andern viele Ger"ate die Wellenform des Stroms, so dass hier weiteres Potential f�r die Unterscheidung verschiedener Ger"ate vorliegt \cite{liang2010load}. \\ %TODO evtl Grafik Spektrum/Wellenform
	 "Micro level" Features bieten sehr gute Unterscheidungsm"oglichkeiten, ben"otigen aber auch ein sehr hochfrequent aufgel"ostes Signal, "ublich sind mehrere kHz, \cite{anderson2012blued} bietet hierzu einen Datensatz mit 12kHz Aufl"osung. \\
	 Messger"ate f�r eine solche Aufl"osung sind wesentlich teurer als die h"aufig f"ur Wirk- und Blindleistung verwendeten Ger"ate mit einer Aufl"osung von 1Hz und sind im Gegensatz zu Letzteren oft nicht in normalen Haushalten mit Smartmeter vorhanden \cite{zeifman2011nonintrusive}. \\ 
	 Daher w"are es w"unschenswert mit den "macro level" Features mit einer Aufl"osung von 1Hz auszukommen.
	 
	 
	

 \include{fazit}

% % % % % % % % % % % % % ANHANG% % % % % % % % % % % % % % % % 
\appendix %Anhang
 \include{anhang} %Anhang

% % % % % % % % % % % % % LITERATUR% % % % % % % % % % % % % % %
\bibliographystyle{geralpha} %Literaturverzeichnis
 \bibliography{Thesis} %Literaturverzeichnis
\end{document}
