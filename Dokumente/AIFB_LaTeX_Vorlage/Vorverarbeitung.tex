\section{Vorverarbeitung}
\label{Vorverarbeitung}


\subsection{Stand der Technik}
\label{Stand der Technik}
	Die Wirkleistung ist das urspr"unglichste Merkmal um die Zust"ande von Ger"aten zu unterscheiden.
	Sie gibt die elektrische Leistung an, die von einem Ger"at in andere Leistungen (z.B. W"arme oder Bewegung) umgewandelt werden kann.  %TODO Verweis finden
	Oft wird die Wirkleistung durch gr"obere Quantisierung oder Normalisierung gegl"attet, um die Erkennung der Klassen zu vereinfachen. \cite{hart1992nonintrusive} verwendet hierzu folgende Formel:\\
	\begin{center}
	$P_{norm} (t) = (\frac{Nennspannung}{V(t)})^2*P(t)$ \\ 
	mit P := Wirkleistung in Watt und V := Spannung in Volt, die Nennspannung betr�gt in Europa 230 Volt.\\
	\end{center}
	Die Normalisierung bietet den Vorteil, dass sie nicht verlustbehaftet ist und wird deshalb h"aufig verwendet, wenn die Spannung gemessen wurde.\\
	Mit der Wirkleistung allein ist man jedoch nicht in der Lage Ger"ate mit sehr "ahnlichem Verbrauch zu unterscheiden, ein Beispiel w"aren hier ein 2kW Motor und ein 2kW Heizelement. 
	 Diese lassen sich mit Hilfe der Blindleistung, welche nicht in andere Leistungen umgewandelt wird, unterscheiden, weil der Motor als induktiver Verbraucher mehr Blindleistung als das Heizelement aus dem Stromnetz zieht. %TODO Grafik Hart und WaMa KIT
	 Nachteil hier ist jedoch, dass die Blindleistung (und auch die Spannung) zus�tzlich gemessen werden muss, die Ber"ucksichtigung dieses Features kostet also gegebenenfalls Geld \cite{zeifman2011nonintrusive}. \\
	 Vergleicht man jedoch eine 60W Gl"uhbirne und einen Laptop mit einem 60W Netzteil, dann stellt man fest, dass diese sich auch unter Hinzunahme der Blindleistung kaum unterscheiden lassen \cite{laughman2003power}. 
	 Hier hilft die Betrachtung von micro level Features. 
	 Dabei werden die Wellenformen und die harmonischen Komponenten des Frequenzspektrums verwendet, um Verbraucher zu unterscheiden. Laughman et al. \cite{laughman2003power} zeigen z.B., dass sich Gl�hbirne und Netzteil in der 3. harmonischen Komponente deutlich unterscheiden.
	 Zus"atzlich ver"andern viele Ger"ate die Wellenform des Stroms, so dass in diesem Bereich weiteres Potential f�r die Unterscheidung verschiedener Ger"ate vorliegt \cite{liang2010load}. \\ %TODO evtl Grafik Spektrum/Wellenform
	 Micro level Features bieten sehr gute Unterscheidungsm"oglichkeiten, ben"otigen aber auch ein sehr hochfrequent aufgel"ostes Signal, "ublich sind mehrere kHz. Anderson et al. \cite{anderson2012blued} bietet einen Datensatz mit 12kHz Aufl"osung. \\
	 Messger"ate f�r eine solche Aufl"osung sind wesentlich teurer als die h"aufig f"ur Wirk- und Blindleistung verwendeten Ger"ate mit einer Aufl"osung von 1Hz und sind im Gegensatz zu Letzteren oft nicht in normalen Haushalten mit Smartmeter vorhanden \cite{zeifman2011nonintrusive}. \\ 
	 Da sie in der Praxis h�ufiger vorhanden sind, wird in dieser Thesis mit den macro level Features mit einer Aufl"osung von 1Hz gearbeitet.
	 
\subsection{Extrahieren der Daten}
\label{Extrahieren der Daten}
	Da die Trainingsdaten leider nur die Wirkleistung zum aktuellen Zeitpunkt angeben, m"ussen diese erneut aus dem gesamten Datensatz herausgezogen werden. Zun"achst wird der Datensatz mit Meter = 7 und Port = 2 gefiltert und so die Waschmaschine ausgew"ahlt. Anschlie�end werden die Startzeiten der einzelnen Waschg"ange "uber ihre Heizphase gesucht und mit Hilfe der L"ange des Waschgangs der zu filternde Abschnitt f�r jedes Profil bestimmt. Hat man diese nun herausgefiltert, hat man neben der Wirkleistung zus"atzlich Spannung, Stromst"arke und aggregierte Wirkleistung zur Verf"ugung. Dies ist wichtig um z.B. die normalisierte Wirkleistung zu berechnen.
	 
\subsection{Zeitreihen}
\label{Zeitreihen}
	Da sich die Klassen der Waschmaschine zum Teil "uberschneiden ist eine fehlerfreie Klassifizierung nur mit Hilfe der Wirkleistung zum aktuellen Zeitpunkt ausgeschlossen. Dies f"allt insbesondere bei Klasse 3 und 4 auf, ihre Flanken fallen oft in den selben Wertebereich, nur ihre Spitzen sind klar unterschiedlich. \\
	Eine M"oglichkeit diese mit einzubeziehen ist die zus"atzliche Verwendung von vergangenen und zuk"unftigen Werten f"ur die Wirkleistung. Diese Arbeit verwendet zun"achst eine Umgebung von 10 Sekunden, also die Werte der Wirkleistung von $t-10$ bis $t+10$ f"ur einen in Sekunden angegeben Zeitpunkt t.\\ 
	Zu beachten ist die spezielle Komprimierung der Daten des ESHL. Zur einfacheren Verarbeitung werden diese zun"achst wieder auf sek"undliche Daten erweitert und anschlie�end die Werte der Wirkleistung der vorherigen und n"achsten 10 Sekunden angeh"angt. Dies erlaubt eine Verarbeitung auch au�erhalb der zeitlichen Reihenfolge mit nur einer einzigen Zeile dieses neuen Datensatzes.\\
	Nachteil ist die n"otige Kenntnis zuk"unftiger Werte, dies verhindert eine live Klassifizierung. Nutzt man den Wert $t+10$ h"angt der Klassifikator der Gegenwart immer mindestens 10 Sekunden nach, dies sollte zur Erstellung gro�er, annotierter Datens"atze jedoch keine Rolle spielen.
	
\subsection{Spezielle Features}
\label{Spezielle Features}
	Neben der M"oglichkeit auf die F"ahigkeit der Generalisierung des KNNs zu vertrauen und diesem einfach die Zeitreihe zu pr"asentieren, besteht auch die M"oglichkeit ihm neben der Wirkleistung zum aktuellen Zeitpunkt noch weitere, speziell f�r die Waschmaschine berechnete Features zu pr"asentieren.\\ 
	F�r die Unterscheidung von Klasse 3 und 4 l"asst sich die Verbrauchspitze in den umgebenden 10 Sekunden berechnen um diese klar trennen zu k"onnen. Um Klasse 4 von Klasse 5 zu trennen kann man sich den Durchschnitt der folgenden Sekunden anschauen, dieser sollte bei einem Punkt der Klasse 5 deutlich h"oher liegen als bei Klasse 4. Die dritte schwierige Unterscheidung ist die von Klasse 1 und 5, hier hilft es sich das Minimum der vorher gehenden Werte anzusehen, Klasse 1 weist hier durch die Pausen Werte von ca. 4 Watt auf, w"ahrend Klasse 5 Werte "uber 200 Watt haben sollte. Alle anderen Klassenunterscheidungen sollten sich mit Hilfe der Wirkleistung treffen lassen.\\
	Zus"atzlich zur Verwendung von Zukunftswerten weist diese Auswahl an Features den Nachteil auf, nur f"ur diese Waschmaschine explizit zu sein, f"ur andere Ger�te m"ussten gegebenenfalls neue Features gefunden werden, dies erh"oht den zeitlichen Aufwand, der zur Klassifizierung eines kompletten Datensatzes n"otig ist, enorm.