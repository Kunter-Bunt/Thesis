\section{Evaluation}
\label{Evaluation}

\subsection{Einleitung}
\label{Einleitung}

\subsection{Tabellensammlung}
\label{Tabellensammlung}
Zu Beginn besch"aftigen wir uns mit der Auswahl des richtigen Zeitfensters. Es zeigt sich schnell, das selbst dieser begrenzte Featureraum bereits mehrere lokale Optima aufweist. Die besten Ergebnisse liefert der Zeitraum von 3 Sekunden um den Zeitpunkt t, f"ur diesen werden weitere Tests durchgef"uhrt.\\

Zun"achst variieren wir die Epochen \textit{N}. Festzustellen ist, dass eine Erh"ohung der Epochen "uber 1000 keine Verbesserung der Genauigkeit mehr mit sich bringt. Die Normalisierung der Wirkleistung verschlechtert das Ergebnis. Dies ist vermutlich auf die vorherige Quantisierung der Daten zur"uckzuf"uhren, diese sorgt daf"ur, dass die Normalisierung die Werte eher streut statt gl"attet. \\
Das Hinzuf"ugen eines zweiten hidden layer f"uhrt zu einer Verbesserung der Klassifikationsgenauigkeit. Die Ver"anderung von Lernrate und Momentum beziehungsweise das Hinzuf"ugen eines dritten hidden layer bringen jedoch keine weiteren Verbesserungen.\\

Die speziell f"ur die Waschmaschine gesuchten Features schneiden schlechter als die Zeitreihen ab. Dies soll kein gro{\ss}er R"uckschlag f"ur diese Arbeit sein, die Zeitreihen sind leichter zu berechnen und besitzen allgemeineren Charakter. Letztere Eigenschaft scheint nicht nur praktisch, sondern auch der Klassifikation dienlich zu sein.\\

Ein genauerer Blick auf die Konfusionsmatrix des Netzes mit dem besten Ergebnis verr"at, dass die meisten Verwechslungen zwischen Klasse 3, 4 und 5 auftreten. Das Netz hat hier offenbar Probleme die Unterschiede in den Verbrauchsspitzen zu erkennen. Diese Art von Fehlern umfasst 62 Instanzen und somit knapp $2/3$ der Fehlklassifikationen. \\
Die 22 Fehler bei denen ein \textit{Off-Zustand} als Klasse 1 klassifiziert wurden sind im Prinzip \textit{off-by-one-Fehler}. Das Netz erkennt nicht die semantische Bedeutung des Features t+0 und klassifiziert auch solche Zeitpunkte als \textit{On-Zustand}, bei denen die Werte t+1,2,3 bzw. t-1,2,3 gr"o{\ss}er Null sind, obwohl t+0 gleich Null ist. Die restlichen Fehlklassifikationen sind "ahnlich zu betrachten, insgesamt machen die \textit{off-by-one-Fehler} mit 33 Instanzen gut $1/3$ der Fehlklassifikationen aus. \\


\begin{table}[p]
\begin{tabular}{l|p{4cm}|p{4cm}|l}
Features & Korrekt klassifizierte Instanzen & Inkorrekt klassifizierte Instanzen & Genauigkeit  \\
\hline
t-10 bis t+10 & 29263 & 740 & 97.5336\% \\
t-9 bis t+9 & 29551 & 452 & 98.4935\% \\
t-8 bis t+8 & 29652 & 351 & 98.8301\% \\
t-7 bis t+7 & 29612 & 391 & 98.6968\% \\
t-6 bis t+6 & 29372 & 631 & 97.8969\% \\
t-5 bis t+5 & 29249 & 754 & 97.4869\% \\
t-4 bis t+4 & 29802 & 201 & 99.3301\% \\
t-3 bis t+3 & 29876 & 127 & 99.5767\% \\
t-2 bis t+2 & 29093 & 910 & 96.967\% \\
t-1 bis t+1 & 26526 & 3477 & 88.4112\% \\
t+0 & 29508 & 495 & 98.3502\% 
\end{tabular}
\caption[Genauigkeit der Zeitreihen Features]{Genauigkeit des Netzes bez"uglich verschiedener L"angen von Zeitreihen, Netz mit Weka Standardkonfiguration (-L 0.3 -M 0.2 -N 500 -H a)}
\label{EvalZeit}
\end{table}

\begin{table}[p]
\begin{tabular}{l|p{1.5cm}|p{3cm}|p{3cm}|l}
Konfiguration & Normali-siert & Korrekt klassifizierte Instanzen & Inkorrekt klassifizierte Instanzen & Genauigkeit  \\
\hline
-L 0.3 -M 0.2 -N 500 -H a & nein & 29876 & 127 & 99.5767\% \\
-L 0.3 -M 0.2 -N 500 -H a & ja & 29870 & 133 & 99.5567\% \\
-L 0.3 -M 0.2 -N 1000 -H a & nein & 29888 & 115 & 99.6167\% \\
-L 0.3 -M 0.2 -N 1000 -H a & ja & 29882 & 121 & 99.5967\% \\
-L 0.3 -M 0.2 -N 1500 -H a & nein & 29883 & 120 & 99.6\% \\
-L 0.3 -M 0.2 -N 1500 -H a & ja & 29870 & 133 & 99.5567\% \\
\hline
-L 0.3 -M 0.2 -N 1000 -H i,o & nein & 29908 & 95 & 99.6834\% \\
-L 0.2 -M 0.2 -N 1000 -H i,o & nein & 29755 & 248 & 99.1734\% \\
-L 0.4 -M 0.2 -N 1000 -H i,o & nein & 29543 & 460 & 98.4668\% \\
-L 0.3 -M 0.1 -N 1000 -H i,o & nein & 29886 & 117 & 99.61\% \\
-L 0.3 -M 0.3 -N 1000 -H i,o & nein & 29540 & 463 & 98.4568\% \\
-L 0.3 -M 0.3 -N 1000 -H i,o,i & nein & 28368 & 1635 & 94.5505\%
\end{tabular}
\caption[Parameterver"anderungen beste Zeitreihe]{Genauigkeit des Netzes bez"uglich verschiedener Parameter des Netzes, als Zeitreihe wird der Zeitraum t-3 bis t+3 verwendet, zus"atzlich wird gegen die normalisierte Wirkleistung getestet}
\label{EvalParam}
\end{table}



\begin{table}[p]
\begin{tabular}{l|p{3cm}|p{3cm}|l}
Konfiguration & Korrekt klassifizierte Instanzen & Inkorrekt klassifizierte Instanzen & Genauigkeit  \\
\hline
-L 0.3 -M 0.2 -N 500 -H a & 29535 & 486 & 98.3811\% \\
-L 0.3 -M 0.2 -N 1000 -H a & 29853 & 168 & 99.4404\% \\
-L 0.3 -M 0.2 -N 1500 -H a & 29535 & 486 & 98.3811\% \\
\end{tabular}
\caption[Spezielle Features]{Genauigkeit des Netzes bez"uglich verschiedener Parameter des Netzes, es werden die speziellen Features verwendet}
\label{EvalExp}
\end{table}

\begin{table}[p]
\begin{tabular}{llllll|l}
a & b & c & d & e & f & $\leftarrow$ classified as  \\
\hline
3670 & 22 & 0 & 0 & 0 & 0 & a = 0  \\
0 & 20984 & 0 & 0 & 0 & 2 & b = 1  \\
0 & 2 & 4762 & 0 & 0 & 0 & c = 2  \\
0 & 3 & 0 & 39 & 15 & 0 & d = 3  \\
0 & 2 & 0 & 18 & 52 & 12 & e = 4  \\
0 & 2 & 0 & 0 & 17 & 401 & f = 5  \\
\end{tabular}
\caption[Konfusionsmatrix]{Konfusionsmatrix des Netzes mit den Features t-3 bis t+3 und den Parametern -L 0.3 -M 0.2 -N 1000 -H i,o}
\label{EvalConf}
\end{table}

