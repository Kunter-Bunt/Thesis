\section{Generierung}
\label{Generierung}
Bei der Generierung lautet die Aufgabe, aus den Klassenzuordnungen eines Zeitabschnitts wieder einen Lastgang zu erstellen. Es ist also die gegenteilige Aufgabe zur Klassifikation. Statt aus einem Lastgang ein Zustandsprofil zu erstellen wird aus einem Zustandsprofil ein Lastgang erstellt.
Zusammen mit der Klassifikation und einer approximierten Zustandsfolge k"onnen so Vorhersagen "uber den Verbrauch der Ger"ate getroffen werden und man ist in der Lage die Nutzung der Ger"ate planen. \\
Die Planung der Ger"atenutzung ist insbesondere in Verbindung mit variablen Energiepreisen wie zum Beispiel dem \textit{Dynamic MeRegio tarif}\cite{FreyEnBW} interessant. Diese Form der Approximation eines Lastgangs greift im Gegensatz zu vorherigen Arbeiten auf die Zustandsfolge zu. Dies sollte zu einer besseren L"osung f"uhren, bringt aber auch das Problem mit sich, dass die Zustandsfolge im Allgemeinen nicht bekannt ist, weil auch die intelligente Waschmaschine nicht mitteilt, welche Zustandsfolge sie einnehmen wird. Das ist aber eine wichtige Information, die dadurch nicht einfach zu erhalten ist. Wir nehmen deshalb an, dass wir bereits eine korrekte Zustandsfolge besitzen. Woher diese kommt l"asst diese Arbeit offen, sie k"onnte z.B. aus einem Algorithmus stammen, der sie vorhersagt oder einer hypothetischen Waschmaschine, die mitteilt, welche Zustandsfolge sie annehmen wird. Dazu reicht es theoretisch, wenn sie mitteilt welches Programm gestartet werden soll. Die Zustandsfolge kann dann in einem einmaligen Test mit der Klassifikation erstellt werden.

\subsection{Stand der Technik}
\label{Stand der Technik}
Das Konzept des L/H-converter aus \cite{hara2008multi} ist ein sehr allgemeines Konzept zum simulieren eines hoch aufgel"osten Signals aus einem niedrig aufgel"osten. Das Zustandsprofil stellt dabei das niedrig aufgel"oste Signal dar, der Wertebereich umfasst die Werte 0,1,2,3,4,5, w"ahrend die Wirkleistung als hoch aufgel"ostes Signal der kompletten Wertebereich von 0 bis 2200 Watt abdeckt. \cite{hara2008multi} gibt f"ur den L/H-converter folgende Formel an.\\
\begin{center}
$x_{high} = R^{L/H} x_{low}$
\end{center}
$R^{L/H}$ stellt dabei die Funktion zur Simulation dar, diese muss in der Implementierung spezifiziert werden.
Abbildung~\ref{converter} zeigt noch einmal die Einordnung des L/H-converter in einem System mit verschiedenen Aufl"osungen, wobei das Low Resolved System in unserem Fall das Zustandsprofil und das High Resolved System den Lastgang repr"asentiert. 
\begin{figure}[h]
\includegraphics[height=0.7\textwidth]{1_Grafiken/multiresolved.jpg}
	\caption[Multi Resolved System]{Schematische Darstellung eines Systems mit verschiedenen Aufl"osungen und den Konvertern, die die Aufl"osung im Informationsfluss anpassen aus \cite{hara2008multi}.}
\label{converter}
\end{figure}

\subsection{Vorgehensweise}
\label{Vorgehensweise}
Die Erstellung eines Lastgangs nur aufgrund des klassifizierten Zustands der Maschine ist nicht besonders sinnvoll, denn das neuronale Netz kann nach dem Training nur die Mittelwerte f"ur die Wirkleistung der einzelnen Klassen angeben und es entsteht ein unrealistischer, extrem glatter Lastgang. Die Verwendung einer Zeitreihe als zus"atzliches Feature schafft einen wesentlich realistischeren Lastgang. Man kann sich hier zu Nutze machen, dass man vor Beginn des Programms implizit annehmen kann, dass sich die Maschine im \textit{Off-Zustand} befindet. Man nimmt also f"ur den ersten zu simulierenden Wert des Waschgangs die vorherige Wirkleistung als 0 an und kann dann iterativ immer einen neuen Wert berechnen. F"ur die Simulation stehen dem Netz die Werte der Wirkleistung von $t-10$ bis $t-1$ und die Klasse von $t+0$ zur Verf"ugung.\\
\begin{center}
$t+0 = F(t-10,...,t-1,K(t+0))$ \\
Beispiel: \\
$1994 \stackrel{!}{=} F(82,82,82,18,18,18,18,18,18,2068, 2)$
\end{center}
Die Funktion $F$ ist die Simulationsfunktion, in diesem Falls also das neuronale Netz. $K(t+0)$ ist die Klassifikation des Zeitpunktes t+0. Die Rekursion von $F$ endet, wenn der Beginn des Waschgangs erreicht ist, dann gilt \\
%\begin{center} 
$t+0 = F(t-1,...,t-10,K(t+0)) = F(0,...,0,K(t+0))\ f"ur\ t = 0$.
%\end{center} 
Bezogen auf die in \ref{Stand der Technik} gegebene Formel f"ur die Simulation gilt: $t+0\ \widehat{=}\ x_{high}\ , K(t+0)\ \widehat{=}\ x_{low}\ und\ F\ \widehat{=}\ R^{L/H}$.