\section{Einleitung}
\label{Einleitung}

\subsection{Motivation}
\label{Motivation}
	Energie zu sparen ist seit langem ein Ziel der Umweltpolitik, dennoch steigt der Energieverbrauch in Industriel"andern kontinuierlich und es werden nach nach wie vor Wege gesucht, diesen zu senken. \\
	Private Haushalte machen 40\% des CO2-Aussto{\ss}es \cite{vandenbergh2008individual} der USA aus und sind deshalb mit im Fokus von Programmen zum Energie sparen. \\
	Je nach Studie besteht f"ur die Haushalte ein Einsparungspotential von bis zu 20\% \cite{armel2013disaggregation}, \cite{fischer2008feedback} untersucht mehrere dieser Studien und beschreibt ein durchschnittliches Einsparungspotential von 5-12\%.\\
	Um dieses Einsparungspotential auszunutzen muss der Nutzer regelm"a{\ss}ig und m"oglichst genau "uber seinen Verbrauch aufgekl"art werden, 	%TODO Graphik Armel
	denn er kann sich oft nicht mit seinem Verbrauch identifizieren, weil dieser unsichtbar und durch die langen Rechnungsintervalle nicht im Kopf des Nutzers pr"asent ist. \\
	Damit der Nutzer anf"angt sein Verhalten zu kontrollieren muss er begreifen, dass sich sein Verhalten auf seinen Verbrauch auswirkt und er ihn auch durch die gezielte Ver"anderungen seines Verhaltens senken kann \cite{fischer2008feedback}.\\
	Non-intrusive load monitoring (NILM) bietet nun die Chance dem Nutzer genau dieses zu geben, denn \cite{kolter2011redd} beschreibt NILM als die Aufgabe aus einem, f"ur den gesamten Haushalt messenden Stromz"ahler, R"uckschl"usse "uber die elektrische Last einzelner Ger"ate zu ziehen. \\
	Ein solches System k"onnte die geforderte regelm"a{\ss}ige und genaue Aufkl"arung des Nutzers über den Verbrauch seiner elektrischen Ger"ate "ubernehmen und ihm so helfen, verschwenderische Verhaltensweisen und Ger"ate zu identifizieren ohne den Haushalt mit vielen digitalen Z"ahlern f"ur die individuellen Ger"ate ausr"usten zu m"ussen, wie es derzeit z.B. mit sogenannten \textit{Energiekostenmessger"aten} auf Steckerbasis praktiziert wird.

\subsection{Fragestellung}
\label{Fragestellung}


\subsection{Weiterer Aufbau}
\label{Weiterer Aufbau}
	Im folgenden wird Kapitel 2 werden die verwendeten Datens"atze beschrieben, insbesondere werden die gemessenen Werte und die daraus berechenbaren Werte untersucht.\\
	Kapitel 3 gibt zun"achst einen "Uberblick "uber die aktuelle Forschung im Bereich Vorverarbeitung und Features, anschlie{\ss}end werden die f"ur diese Arbeit verwendeten Vorverarbeitungsschritte und Features erl"autert. \\
	In Kapitel 4 soll schlie{\ss}lich die eigentliche Klassifizierung beschrieben werden, hier werden insbesondere das verwendete Netz und Trainingsmethoden besprochen. Auch hier wird ein kurzer "Uberblick "uber die aktuelle Forschung gegeben.\\
	Die Evaluierung der Klassifikation findet in Kapitel 5 statt. Hier werden verschiedene Klassifizierungsaufgaben ausgef"uhrt und ausgewertet.\\
	Am Schluss steht ein Fazit, in dem die Ergebnisse zusammengefasst und R"uckbezug auf die urspr"ungliche Fragestellung genommen wird.\\