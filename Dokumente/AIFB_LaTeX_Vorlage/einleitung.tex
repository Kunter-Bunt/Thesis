\section{Einleitung}
\label{Einleitung}

\subsection{Motivation}
\label{Motivation}
	Energie zu sparen ist seit Langem ein Ziel der Umweltpolitik, dennoch steigt der Energieverbrauch in Industriel"andern kontinuierlich und es werden nach nach wie vor Wege gesucht, diesen zu senken. 
	In den USA verursachen private Haushalte 40\% des CO2-Aussto{\ss}es \cite{vandenbergh2008individual} und stehen deshalb im Fokus vieler Programmen zum Energie Sparen. \\
	Je nach Studie besteht f"ur die Haushalte ein Einsparungspotential von bis zu 20\% \cite{armel2013disaggregation}, Fischer \cite{fischer2008feedback} untersucht mehrere dieser Studien und beschreibt ein durchschnittliches Einsparungspotential von 5-12\%. 
	Um dieses Einsparungspotential auszunutzen, muss der Nutzer regelm"a{\ss}ig und m"oglichst genau "uber seinen Verbrauch aufgekl"art werden. 	%TODO Graphik Armel
	Er kann sich oft nicht mit seinem Verbrauch identifizieren, weil dieser unsichtbar und durch die langen Rechnungsintervalle nicht pr"asent ist. \\
	Damit der Nutzer anf"angt sich selbst zu kontrollieren, muss er begreifen, dass sich sein Verhalten auf seinen Verbrauch auswirkt und er ihn auch durch die gezielte Ver"anderungen seines Handelns senken kann \cite{fischer2008feedback}.\\
	Non-intrusive load monitoring (NILM) bietet nun die Chance dem Nutzer eine regelm"a{\ss}ige, detaillierte R"uckmeldung zu seinem Energiekonsum zu geben, denn Kolter \cite{kolter2011redd} beschreibt NILM als die Aufgabe aus einem, f"ur den gesamten Haushalt messenden Stromz"ahler, R"uckschl"usse "uber die elektrische Last einzelner Ger"ate zu ziehen.
	Ein solches System k"onnte dem Konsumenten helfen, verschwenderische Verhaltensweisen und Ger"ate zu identifizieren ohne den Haushalt mit vielen digitalen Z"ahlern f"ur die individuellen Ger"ate ausr"usten zu m"ussen, wie es derzeit z.B. mit sogenannten \textit{Energiekostenmessger"aten} auf Steckerbasis praktiziert wird.

\subsection{Fragestellung}
\label{Fragestellung}


\subsection{Weiterer Aufbau}
\label{Weiterer Aufbau}
	Im folgenden Kapitel 2 werden die verwendeten, disaggregierten Energiedatens"atze beschrieben. Insbesondere werden die gemessenen Werte und die daraus berechenbaren Werte untersucht. 
	Kapitel 3 gibt zun"achst einen "Uberblick "uber die aktuelle Forschung im Bereich Vorverarbeitung der Daten und Feature-Auswahl, anschlie{\ss}end werden die f"ur diese Arbeit verwendeten Vorverarbeitungsschritte und Features erl"autert.
	In Kapitel 4 soll schlie{\ss}lich die eigentliche Klassifizierung beschrieben werden, hier werden insbesondere das verwendete Netz und die Trainingsmethoden besprochen. Auch hier wird es einen kurzen "Uberblick "uber die aktuelle Forschung geben.
	Die Evaluierung der Klassifikation findet in Kapitel 5 statt. Hier werden verschiedene Klassifizierungsaufgaben ausgef"uhrt und ausgewertet. 
	Am Schluss steht ein Fazit, in dem die Ergebnisse zusammengefasst und R"uckbezug auf die urspr"ungliche Fragestellung genommen wird. 