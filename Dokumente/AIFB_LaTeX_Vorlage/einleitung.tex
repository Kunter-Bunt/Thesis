\section{Einleitung}
\label{Einleitung}

\subsection{Motivation}
\label{Motivation}

Energie zu sparen ist ein wichtiges Ziel der Umweltpolitik, private Haushalte stehen hier mit im Fokus, weil sie  40\% des CO2-Aussto{\ss}es \cite{vandenbergh2008individual} der USA ausmachen. \\
Je nach Studie besteht für die Haushalte ein Einsparungspotential von bis zu 20\% \cite{armel2013disaggregation}, \cite{fischer2008feedback} untersucht mehrere dieser Studien und beschreibt ein durchschnittliches Einsparungspotential von 5-12\%.\\
Um dieses Einsparungspotential auszunutzen muss der Nutzer regelm"a{\ss}ig und m"oglichst genau "uber seinen Verbrauch aufgekl"art werden, %TODO Graphik Armel
denn er kann sich oft nicht mit seinem Verbrauch identifizieren, weil dieser unsichtbar und durch die langen Rechnungsintervalle nicht im Kopf des Nutzers pr"asent ist. \\
Damit der Nutzer anf"angt sein Verhalten zu kontrollieren muss er begreifen, dass sich sein Verhalten auf seinen Verbrauch auswirkt und er ihn auch durch die gezielte Ver"anderungen seines Verhaltens senken kann.\cite{fischer2008feedback}\\
Non-intrusive load monitoring (NILM) bietet nun die Chance dem Nutzer genau dieses zu geben, nach \cite{kolter2011redd} ist NILM die Aufgabe aus einem, für den gesamten Haushalt messenden, Stromz"ahler Rückschlüsse über die elektrische Last einzelner Ger"ate zu ziehen. \\
Ein solches System k"onnte die geforderte regelm"a{\ss}ige und genaue Aufkl"arung des Nutzers "ubernehmen und ihm so helfen, verschwenderische Verhaltensweisen und Ger"ate zu identifizieren ohne den Haushalt mit vielen digitalen Z"ahlern für die individuellen Ger"ate ausr"usten zu m"ussen.

\subsection{Fragestellung}
\label{Fragestellung}


\subsection{Vorgehensweise}
\label{Vorgehensweise}
